\documentclass{article}

\usepackage{amsmath} % assumes amsmath package installed
\usepackage{hyperref}

\title{Music with Python}
\author{Alexis Mifsud}


\begin{document}
\maketitle

The aim of this paper is to group all the scientific part and conventions of the music\_python package.

\tableofcontents

%\newpage

\section{Signal from spectrum}
\renewcommand\vec[1]{\mathbf{#1}}
\newcommand\s{\vec{s}}
\newcommand\br{\vec{b}}
\newcommand\p{\vec{p}}

\subsection{Fourier series}

Let define $\s$ the spectrum of the desired signal. In python the spectrum is a list of sub-list where each sub-list contain the frequency and the relative amplitude to the fondamental. For the harmonic i:
\begin{equation}
\s[i]=[f_i,A_i]
\end{equation}
where $f_i$ is the frequency of the harmonic and $A_i$ is the amplitude of the harmonic.

We have for each time k:
\begin{equation}
y_k=\sum_{i=1}^{N} A_i sin(2 \pi f_i k)
\end{equation}
where N is the number of harmonics

\subsection{Time representation}

In numeric music, the time is discretized with a period called the \textit{bitrate} in Hertz which will be noted $\br$. Usually $\br=44100 Hz$

\section{Keeping noise power when adding harmonics}

\subsubsection*{For one harmonic (the fondamental):}

\begin{equation}
\s = \begin{bmatrix}
f_1 & A_1
\end{bmatrix}
=
\begin{bmatrix}
f_1 & 1
\end{bmatrix}
\end{equation}
By convention, we want to keep: $-1<A<1$. This permit to encode the amplitude like:
\begin{equation}
char(A)=A*127+128
\end{equation}
where char(A) is a character ($0<char(A)<255$)

\subsubsection*{For more harmonics}

The problem is that the signal power as to be keep. Let's consider the folowing case where we add $f_2$ and $f_3$:
\begin{align}
\s
= \begin{bmatrix}
f_1 & A_1 \\
f_2 & A_2 \\
f_3 & A_3 \\
\end{bmatrix}
= \begin{bmatrix}
f_1 & 1 \\
f_2 & 0.5 \\
f_3 & 0.25 \\
\end{bmatrix} \nonumber \\ 
\end{align}
where 0.5 mean 50\% in amplitude of f1.

In this case we have the followinf power:
\begin{equation}
\p = \sum_{i=1}^3 A_i^2 = 0.605 W/m^2
\end{equation}

If we want to have the same power:
\begin{align}
\sum_{i=1}^N A_i^2 =1 \\
A_1^2 (1+\sum_{i=2}^N A_i^2) =1 \\
\end{align}
So:
\begin{align}
A'_1=\frac{1}{\sqrt{1+\sum_{i=2}^N A_i^2}} \\
A'_i = A'_1  A_i
\end{align}

Finally I removed all the squared values...

\section{Interresting websites}

\begin{itemize}
\item \url{http://musique.ac-dijon.fr/bac2001/grisey/pages/acoustic.htm}
\item \url{http://ssl7.ovh.net/~pianoteq/philippe/sons-musique.pdf}
\item \url{https://fr.wikipedia.org/wiki/LilyPond}
\end{itemize}











\end{document}